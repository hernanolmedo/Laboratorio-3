\documentclass{memoria}

\begin{document}
	\portada{Laboratorio 3 Organización de Computadores: Pipeline}{Hernán Olmedo}{Profesores: \\Felipe Garay,\\Erika Rosas,\\Nicolás Hidalgo\\Ayudante: \\Ian Mejias}{\today}
	\indices
	
	\capitulo{Introducción}{\bigskip\bigskip
		Hoy en día se vive en un mundo globalizado, donde la información se propaga rápidamente por el mundo en todo ámbito, ya sea industrial, redes sociales, economía, investigación, etc. La informática juega un papel crucial, siendo la ciencia encargada de proveer los medios a través de los cuales se propaga y procesa dicha información, todas las actividades ligadas al uso de medios informáticos exigen la mayor rapidez de respuesta posible. Cada día estas exigencias van creciendo, aumentando la cantidad de datos a procesar y en el menor tiempo posible. Una de las herramientas de las que disponen los informáticos para aumentar la rapidez de respuesta en sus programas es el profiling o perfilaje que nos entrega información detallada del tiempo que está tomando ejecutarse un programa. En este laboratorio se utilizará esta herramienta para reducir el tiempo de ejecución de una función que calcula la tangente hiperbólica mediante el análisis del pipeline.\bigskip
		El presente laboratorio requiere la implementación de la función tanh utilizando la serie de Taylor, se pide hacerlo en lenguaje C y con 3 decimales de precisión. Una vez implementada, se dibuja el pipeline identificando posibles hazards y posterior a esto se itera modificando el código hasta que ya no se encuentren hazards. En cada iteración se debe utilizar gprof, un profiler para sistemas operativos GNU y con este se debe calcular los tiempos que van tomando los distintos códigos para luego sacar conclusiones respecto a esto.\bigskip
		El código será realizado en Ubuntu con el compilador GCC. El valor al cual se le va a calcular la tangente hiperbólica se debe introducir como argumento del ejecutable con la bandera "-n" espacio y el valor. Para manipular este argumento dentro del código se hace uso de la función getopt que facilita el análisis de los argumentos e identifica los errores al momento de ingresarlos. Además el informe es desarrollado en Latex y se utiliza un repositorio en Github para registrar los avances realizados.\bigskip
		Al realizar este laboratorio se comprueba empíricamente, como afecta en el tiempo de ejecución el orden de las instrucciones en un código y las dependencias que se generan entre estas. Así se podrá tener un criterio para futuros proyectos a la hora programar, con el fin de reducir lo más posible el tiempo de ejecución.\bigskip
		En el capítulo 2 se habla sobre el pipeline, sus etapas en el procesador MIPS y sobre los hazard. En el capítulo 3 se explica como fue llevado a cabo el desarrollo del trabajo. En el capítulo 4 se discuten los resultados obtenidos y se explican de acuerdo a la materia vista en clases y a investigación personal.
		}
	
	\capitulo{Marco Teórico}{\bigskip\bigskip%sobre el pipeline, etapas del pipeline de mips y hazards
			El pipeline es un técnica de implementación en la cual se traslapan las instrucciones durante la ejecución.\bigskip
		\seccion{Etapas del Pipeline}{\bigskip
			En MIPS se tienen 5 etapas:\bigskip\bigskip
			IF: En esta etapa se busca la instrucción en memoria.\bigskip
			ID: Se leen los registros mientras se decodifica la instrucción.\bigskip
			EX: Se ejecuta la operación de la instrucción o bien se calcula una dirección de memoria.\bigskip
			MEM: En esta etapa se accede a un operando que se encuentra en memoria.\bigskip
			WB: Finalmente se escriben los resultados en un registro.\bigskip\bigskip
			}
		\seccion{Hazards}{\bigskip
			Los hazards son obstáculos que se producen durante la ejecución del pipeline cuando una instrucción no se puede ejecutar en el siguiente ciclo de reloj. Existen 3 tipos de hazards:\bigskip\bigskip
			Hazard Estructural: Estos se producen cuando el hardware no soporta una combinación de instrucciones.\bigskip
			Hazard de Datos: Se producen cuando los datos que requiere una instrucción aún no están disponibles.\bigskip
			Hazard de Control: Se producen cuando la instrucción buscada no es la que se tiene que ejecutar.\bigskip\bigskip
			Forwarding: Técnica que permite resolver hazard de datos mediante la utilización de buffers. 
			}					
		}
	
	\capitulo{Desarrollo}{\bigskip\bigskip
	
		}
	\capitulo{Discusiones de los resultados}{\bigskip\bigskip
	% q hace cada salida, leer sobre resultados de gprof
	}

	\capitulo{Conclusión}{\bigskip\bigskip\bigskip
		}
	\nocite{enunciado}
	\nocite{profiling}
	\nocite{repo}
	
	\bibliografiasincita
	
	
\end{document}
