\documentclass{memoria}


\begin{document}
	\portada{Laboratorio 2 Organización de Computadores: Caché}{Hernán Olmedo}{Profesores: \\Felipe Garay,\\Erika Rosas,\\Nicolás Hidalgo\\Ayudante: \\Ian Mejias}{\today}
	\indices
	
	\capitulo{Introducción}{\bigskip\bigskip
		Los computadores hoy en día juegan un papel preponderante en el desarrollo tecnológico y pese a su rápida evolución, las exigencias no se quedan atrás con el paso del tiempo y cada vez son más las áreas que exigen procesar grandes cantidades de datos en un tiempo reducido. Por ejemplo procesar los datos de una sonda espacial, áreas como las finanzas, la edición de gráficos, entre otras. Una de las partes claves en la rapidez de respuesta de un computador es el manejo de la memoria, en este informe trataremos la memoria de acceso rápido llamada caché, las configuraciones que esta puede tener y como esto influye en la rapidez del computador. \cite{enunciado} \bigskip
		
		En este laboratorio se solicita implementar 2 algoritmos de ordenamiento, uno iterativo y otro recursivo a elección. Estos deben ser implementados en lenguaje ensamblador para el procesador MIPS utilizando el simulador MARS, cada algoritmo constituye un programa y cada programa debe ser capaz de abrir un archivo de texto que contiene un número por línea, pasar estos números a memoria, ordenarlos y finalmente crear un nuevo archivo de texto con los números ordenados. Utilizando la herramienta Data Cache Simulator de MARS se deben probar las distintas configuraciones de caché y analizar los resultados.\bigskip
		
		Los algoritmos a implementar serán el Quicksort como recursivo y el Bubble Sort como iterativo. Para la implementación de estos algoritmos se utilizó el stack de MIPS para ordenar allí los números. Utilizando la herramienta Data Cache Simulator se obtienen los resultados que serán analizados en este informe.\bigskip
		
		Con la realización de este trabajo se busca comprobar empíricamente la materia vista en clases respecto a las configuraciones del cache, además de mejorar nuestras habilidades para programar en un lenguaje de muy bajo nivel como lo es el del procesador MIPS.\bigskip
		
		En el capítulo 2 se describen las configuraciones que puede tener el caché así como también conceptos claves en el entendimiento del cache y los algoritmos de ordenamiento. En el capítulo 3 se explica como fue llevado a cabo el desarrollo del trabajo. En el capítulo 4 se discuten los resultados obtenidos y se explican de acuerdo a la materia vista en clases y a investigación personal
		}
	
	\capitulo{Marco Teórico}{\bigskip\bigskip%sobre el pipeline, etapas del pipeline de mips y hazards
		El pipeline es un técnica de implementación en la cual se traslapan las instrucciones durante la ejecución.\bigskip
		\seccion{Etapas del Pipeline}{\bigskip
			En MIPS se tienen 5 etapas:\bigskip\bigskip
			IF: En esta etapa se busca la instrucción en memoria.\bigskip\bigskip
			ID: Se leen los registros mientras se decodifica la instrucción.\bigskip\bigskip
			EX: Se ejecuta la operación de la instrucción o bien se calcula una dirección de memoria.\bigskip\bigskip
			MEM: En esta etapa se accede a un operando que se encuentra en memoria.\bigskip\bigskip
			WB: Finalmente se escriben los resultados en un registro.\bigskip\bigskip

			\bigskip\bigskip
			}
		\seccion{Hazard}{\bigskip
			Los hazard son obstáculos que se producen durante la ejecución del pipeline cuando una instrucción no se puede ejecutar en el siguiente ciclo de reloj. Existen 3 tipos de hazard:\bigskip\bigskip
			Hazard estructurales: Se producen cuando el hardware no puede soportar una combinación de instrucciones en un ciclo de reloj.\bigskip\bigskip
			Hazard de datos: Se producen cuando una instrucción requiere de datos que aún no están disponibles.\bigskip\bigskip
			Hazard de control: Se producen cuando se busca una instrucción pero esta no era la que se necesitaba ejecutar, el flujo  de ejecución de las instrucciones cambia.\bigskip\bigskip
			\bigskip\bigskip
			}
			Forwarding: Método para resolver hazard de datos que mediante la utilización de buffers puede hacer uso de los datos sin  esperar a que esten en los registros o en memoria.
		}
	
	\capitulo{Desarrollo}{\bigskip\bigskip
	
		}
	\capitulo{Discusiones de los resultados}{\bigskip\bigskip
	}

	\capitulo{Conclusión}{\bigskip\bigskip\bigskip
		}

	\bibliografiasincita
	
	
\end{document}
